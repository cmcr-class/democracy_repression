\documentclass[ignorenonframetext,]{beamer}
\setbeamertemplate{caption}[numbered]
\setbeamertemplate{caption label separator}{: }
\setbeamercolor{caption name}{fg=normal text.fg}
\beamertemplatenavigationsymbolsempty
\usepackage{lmodern}
\usepackage{amssymb,amsmath}
\usepackage{ifxetex,ifluatex}
\usepackage{fixltx2e} % provides \textsubscript
\ifnum 0\ifxetex 1\fi\ifluatex 1\fi=0 % if pdftex
  \usepackage[T1]{fontenc}
  \usepackage[utf8]{inputenc}
\else % if luatex or xelatex
  \ifxetex
    \usepackage{mathspec}
  \else
    \usepackage{fontspec}
  \fi
  \defaultfontfeatures{Ligatures=TeX,Scale=MatchLowercase}
\fi
% use upquote if available, for straight quotes in verbatim environments
\IfFileExists{upquote.sty}{\usepackage{upquote}}{}
% use microtype if available
\IfFileExists{microtype.sty}{%
\usepackage{microtype}
\UseMicrotypeSet[protrusion]{basicmath} % disable protrusion for tt fonts
}{}
\newif\ifbibliography
\hypersetup{
            pdftitle={Democratic Repression: Responding in Kind?},
            pdfauthor={Christopher Junk},
            pdfborder={0 0 0},
            breaklinks=true}
\urlstyle{same}  % don't use monospace font for urls

% Prevent slide breaks in the middle of a paragraph:
\widowpenalties 1 10000
\raggedbottom

\AtBeginPart{
  \let\insertpartnumber\relax
  \let\partname\relax
  \frame{\partpage}
}
\AtBeginSection{
  \ifbibliography
  \else
    \let\insertsectionnumber\relax
    \let\sectionname\relax
    \frame{\sectionpage}
  \fi
}
\AtBeginSubsection{
  \let\insertsubsectionnumber\relax
  \let\subsectionname\relax
  \frame{\subsectionpage}
}

\setlength{\parindent}{0pt}
\setlength{\parskip}{6pt plus 2pt minus 1pt}
\setlength{\emergencystretch}{3em}  % prevent overfull lines
\providecommand{\tightlist}{%
  \setlength{\itemsep}{0pt}\setlength{\parskip}{0pt}}
\setcounter{secnumdepth}{0}

\title{Democratic Repression: Responding in Kind?}
\author{Christopher Junk}
\date{March 10, 2019}

\begin{document}
\frame{\titlepage}

\begin{frame}[fragile]

The variables that I care about are the following:

\begin{itemize}
\tightlist
\item
  start\_year\_month: year-month variable for start of event\\
\item
  end\_year\_month: year-month variable for end of event\\
\item
  duration: number of months between start and end\\
\item
  cowcode: COW country codes\\
\item
  country: country text\\
\item
  pol\_express: symbolic expression\\
\item
  mass\_express: mass protests\\
\item
  pol\_viol: political violence\\
\item
  intang\_rep: intangible repression\\
\item
  st\_repress: tangible repression\\
\item
  stat\_viol: violent repression
\end{itemize}

e and Temporal Differences When do democracies resort to repression? The
literature on the dissent-repression nex Dissent and Repression: Regimus
has been active for over forty years, but there are still important
methodological improvements that can be made to this literature. To
date, the most concrete findings are summarized in Davenport's (2007)
literature review: the law of coercive responsiveness, more murder in
the middle, and the domestic democratic peace. The law of coercive
responsiveness is that states are most likely to repress in response to
a threat. More murder in the middle is the finding that regime types
somewhere between full authoritarian and full democracy are the most
violent repressors. The domestic democratic peace is threefold:
democratic institutions increase the costs of state repression,
democratic citizens accept democratic values of nonviolent expression
and negotiating, and democracies provide alternative measures for
political control. It is important to note that the domestic democratic
peace only applies to states that cross a certain threshold of
democratic-ness at which point they become `full' democracies (Davenport
1995, 1999; Davenport and Armstrong 2004; Poe and Tate 1994; Regan and
Henderson 2002). Further motivation comes from the terrorism literature.
A variety of studies have shown that as democratic citizens perceive
their own threat levels increasing, they offer up individual autonomy
and rights in exchange for greater security offered by the state
(Bloch-Elkon 2011; Davis and Silver 2004; Finseraas and Listhaug 2013;
Mondak and Hurwitz 2012). Importantly, this shows that in democracies if
the dissent sufficiently threatens the population at large they will
accept repression if it means lower perceived threat levels. This
mechanism provides legitimacy for the use of repressive policy by
democratic states. I define dissent broadly: any action that is
political in motivation initiated by a non-government entity with the
intention of political change. In measurement, I divide dissent into
three categories that are captured through three distinct independent
variables: symbolic dissent, mass dissent, and political violence.
Symbolic dissent requires no mass mobilization and can take the form of
leaflets, books, films and other mediums that criticize the state. Mass
dissent is nonviolent mobilization of large groups of people. Obvious
examples of this include protests such as the Women's March and other
such mass protests. Political violence is the use of violence to create
political change. This definition improves upon recent literature by
including a broader conception of dissent and differentiating between
different them. Davenport (2007, 2) defines repression as: ``the actual
or threatened use of physical sanctions against an individual or
organization, within the territorial jurisdiction of the state, for the
purpose of imposing a cost on the target as well as deterring specific
activities and/or beliefs perceived to be challenging to government
personnel, practices or institutions.'' I expand this definition of
repression in an important way: the negation of civil liberties such as
freedom of speech, religion, association, gathering, and movement in
nonviolent ways also constitutes repression. The literature mostly
ignores this type of repression, largely due to data issues. Much of the
repression datasets that exist are based on and developed around violent
repression of physical integrity rights (Francisco 1996; Gibney et al.
2018; Poe and Tate 1994), or only take into account human rights
violations or scores (Davenport 1999; Davenport and Armstrong 2004;
Franklin 2008; Regan and Henderson 2002; Stohl et al. 1986). I
categorize repression into three types: intangible state repression,
tangible state repression, and violent repression. Intangible state
repression violates basic rights in a nonviolent manner such as
censorship, martial law, imposing curfews, etc. Tangible state
repression involves the use of state capacity to coerce citizens without
using violence, detention, or the military. Violent repression uses
lethal force, the military, detention, torture, or any combination of
violent measures taken to coerce individuals. These definitions are
provided by the Social, Political, Economic Events Database (SPEED)
which is the data used in this project (Nardulli, Althaus, and Hayes
2015). Researchers so far have considered only contemporary or directly
previous time periods when considering the effect of dissent on state
repression with few exceptions(Davenport 1996) . By considering only the
immediate temporal proximity, the previous analyses have underestimated
the effect of residual dissent in state memory. I will show that when
the analysis takes into account the memory of dissent democracies
repress in response to persistent recent threats. I will show that
states repress as a reaction to the memory of dissent because that
represents the full threat experienced by a state, as well as in
response to different types of dissent. When dissidents take to the
street, any actions that they take in isolation are weak relative to
state control and power. However, when the recent history of dissent is
considered, the state looks at a broader pattern of dissent and updates
its perceived threat level. As dissent becomes more severe in recent
memory, the threat level rises to both the state and the population at
large and repression becomes a more attractive tool for political
control. Another improvement I make in this paper is a more appropriate
temporal unit of analysis. Most existing studies have used country-year
units of analysis. Repression and dissent are both day-to-day activities
that can vary greatly in intensity and frequency within a year.
Aggregating both measures to yearly levels loses large amounts of
variation in the dependent and explanatory variables. I use SPEED to
generate a country-month unit of analysis that increases the observable
temporal variance. This also allows a more fine-grained analysis of how
quickly a state responds to dissent. One final improvement I will make
on the literature is improving the consideration of democracy as a
variable. Whereas most of the literature to date has broadly used the
Polity series of variables as a democratic indicator (Davenport 1999;
Henderson 1991; Poe and Tate 1994), I will use more explicit definitions
of democracy. I define democracy according to its electoral
institutions. As such, I will use the Lexical Index, which is an
additive scale considering electoral components of a state. I define a
full democracy according to a six out of six on the Lexical Index:
elections exist, elections allow multiparty competition, legislative and
executive offices are filled through election, and full female and male
suffrage. I find that democracies repress as a response to symbolic
dissent and as a response to political violence, but not as a response
to mass dissent. This suggests that democracies will repress if they
think it will go unnoticed, or if they think it is legitimate. Dissent,
Repression, and Response Before discussing how the state responds to
dissent, it is important to note what factors predict state repression
in general. There is a rich literature theorizing and testing different
explanatory variables that influence repression without considering it
as a response to dissent. Mitchell and McCormick (1988) provide some of
the earliest quantitative analysis. Their work suggests that state
wealth has at least a moderately negative impact on the probability of a
state repressing. Interestingly, they find that states with British
colonial heritage are the least likely to repress because of the
institutions inherited that make abuse of state powers taboo. Mitchell
and McCormick also show evidence that the most autocratic rulers
(e.g.~personalist and totalitarian) are the most likely to violate human
rights. Henderson (1991) shows that democracy and economic growth
negatively predict repression, while inequality positively predicts
repression. This is the first appearance of the democratic argument in
empirical testing. Poe and Tate (1994) corroborate this finding but
suggest that economic standing alone is enough to dissuade state
repression. Davenport (1995) also finds support for the hypothesis that
democracies repress less, and he uses Taylor and Jodice (1983) measures
of repression that focus on nonviolent repression. Davenport and
Armstrong (2004) employ nonparametric testing in an atheoretical
examination to show that states must hit a certain threshold of
democratic-ness before repression starts to decline. This is the
emergence of the ``more murder in the middle'' (Davenport 2007, 11)
pattern: states at the extreme ends of regime types have no need for
repression, but states in between the extremes face different obstacles
that makes repression more likely. This is because full autocracies have
consolidated power and little respect for human rights initially, and
thus are unhindered in their use of repression as a policy tool. This
creates an expectation that repression would be a standard response to
dissent, and thus dissent is less likely, and repression is less
necessary. In full democracies repression is theoretically an
unacceptable policy tool, while (nonviolent) dissent is a protected
civil liberty. Therefore, dissent should not beget repression both
because it is legal and because the state's use of repression would be
illegitimate. A last variable influencing state repression independent
of dissent is whether a state is involved in a war. Poe and Tate (1994)
find that wars increase repression domestically because a state needs to
maintain political control more during turbulent political times.
Danneman and Ritter (2014) show that states recognize the contagion
threat of neighboring civil wars and preemptively repress their own
population in order to maintain power and quell potential rebellion.
Davenport (1995) suggests that a state's threat perception is a function
of three dissent factors: frequency, severity, and variety. As these
three factors increase, states perceive threat as being greater and
become more likely to repress. They also show that as levels of dissent
differ from the norm (average level of dissent), states perceive a
greater threat and repress more often. Regan and Henderson (2002)
corroborate this finding broadly showing that states that feel
threatened now are more likely to repress. Davenport (1996) later shows
that lagged yearly sums of political conflict positively associate with
repression, but he does not test for moving sums or moving averages.
Moore (2000) shows that states change their strategy for dealing with
dissent (accommodation or repression) when the current strategy is met
with further dissent. The relationship between dissidents and repressors
is also likely endogenous. Lichbach (1987) developed a model of
dissident response to repression. In this model, dissidents have two
dissent options: legal (nonviolent) and illegal (violent). Dissidents
prefer to use more effective dissent strategies and will invest more
time and money into those strategies. When states recognize which
strategies are more successful (gain more accommodation) states will
selectively repress to lower the efficacy of dissident movements. This
model suggests that when states begin to repress strategically, overall
levels of dissent increase as dissidents increase quantity of dissent to
make up for lost quality of dissent. Moore (1998) tests this model
against other theorized relationships and finds support for exclusively
Lichbach's argument. Francisco (1995, 1996) compares all theorized
dissent responses to repression. He identifies five theorized
relationships: backlash, inverted-U, nonlinear, and adaptive. A backlash
relationship predicts an absolute increase in dissent as a response to
repression. An inverted-U response posits that at high and low levels of
repression dissent is uncommon, but the middle-ground sees high levels
of dissent. The nonlinear relationship suggests that repression and
dissent oscillate in a relatively unpredictable manner. The adaptive
dynamic is Lichbach's pattern. Francisco also finds comparatively more
evidence for Lichbach's dynamic theory. The literature has not yet
utilized computer coded event datasets to test the dynamics of the
dissent-repression nexus. Moreover, no work to date has provided a
comprehensive long-run analysis across all states in the post-WWII
period. The SPEED dataset allows me to do that. Datasets used in the
past have been limited to least developed countries (Regan and Henderson
2002), violent dissent types (Davenport 1995; Francisco 1996; Moore
1998, 2000; Regan and Henderson 2002), and yearly aggregations
(Davenport 1996; Moore 1998, 2000) or single-year estimations (Henderson
1991, 1993). Additionally, all previous work is more constrained in its
temporal domain than this analysis. These all contribute in important
ways to the basis of the theory proposed in the next section. I intend
to expand upon their theory and use better fine-grained data to estimate
my models. State Memory of Dissent and the Use of Repression This theory
is based on two fundamentally assumed goals of leaders that do not vary
across regime types: political survival(Bueno de Mesquita et al. 2003),
and the provision of security for the state. All leaders today want to
be leaders tomorrow, and political survival is of the utmost importance.
The manner in which political survival is secured can vary greatly
across regime types from the co-optation, legitimation, and repression
in autocracies (Gerschewski 2013) to electoral success in democracies.
The provision of security can be broadly interpreted as the provision of
any necessary public goods to prevent the state from descending into
anarchic chaos of continual violence. This assumption is intentionally
broad to show that a leader must provide, at minimum, security
sufficient to maintain territorial integrity, and at best, prosperity
and high standards of living. It is important to note that security is
not strictly a public good because it is potentially excludable from
certain populations (e.g.~selective repression of minority groups).
Security defined as maintaining the territorial integrity of the state
(i.e.~avoiding civil war and state collapse), however, is
non-excludable. This type of security provision is particularly salient
to the theoretical understanding of repression in democracies because
without territorial integrity the leadership of the state has no
governing authority. Once dissent presents a sufficient threat to the
security of the state defined this way the state can legitimately
repress or violate civil liberties in order to provide security. There
is ample evidence in the terrorism literature that in democracies
leadership even has support for such actions (Bloch-Elkon 2011; Davis
and Silver 2004; Mondak and Hurwitz 2012). According to modern
democratic theory, democracies are committed to civil liberties and
human dignity rights in a broad sense (Dahl 1989; Ober 2012). Power in
democracies is derived through legitimate electoral success and popular
support. Democratic norms of nonviolent bargaining differ starkly from
the less institutionalized bargaining practices of autocracies (Haggard
and Kaufman 2016). Because of these norms, policymaking in democratic
states consists of policy changes to provide sufficient public and
private goods to gain support of a winning coalition. Leadership which
provides unsatisfactory policy to accrue sufficient support to create a
winning coalition will lose upcoming elections and be replaced. Leaders
must respond to dissent with policy in ways that provide sufficient
security to please their winning coalition. Given that this paper
considers only full democracies, this winning coalition is at least half
the population. In a majoritarian system like the US the president must
please the at least half of the voters to keep his seat. In a
proportional representation system like the UK the policy of the prime
minister must please his/her own party as well as any parties also a
member of a coalition government. Policy made by incumbents must not
contradict the basis of democratic rule: respect for civil liberties and
human dignity rights. However, leaders are also afforded some leeway to
act because of democratic legitimacy: leaders won their power through
free and fair electoral processes and are therefore entitled to some
autonomy in responding to threats as they see fit. They must respond in
ways that do not compromise their chances of future political success.
In an optimistic view of democratic governance leaders may only resort
to repression once the level of threat is sufficiently high that
repressive policy is the best way to assure security for the state and
satisfy the winning coalition. The difficulty of traditional definitions
of democracies as commitments to human rights is that these states will
by definition never repress, even nonviolently. I chose to define
democracies in terms of electoral institutions because it largely
escapes the tautology just described and allows for states that are
electorally democratic to be logically capable of repressing in the
coding scheme. This definition and operationalization still assumes that
elections are free and suffrage is universal, but it makes no
assumptions about state behavior beyond their behavior around elections.
When dissidents dissent, political survival and state security are both
potentially threatened. Threat has been previously theorized to be a
function of: quantity of dissent, number of strategy types, severity,
and difference from the norm (Davenport 1995). I distinguish between
three main types of dissent: symbolic ``small-gauge'' dissent, mass
dissent, and political violence. Symbolic dissent is nonviolent non-mass
dissent that express political discontent. Mass dissent is also
nonviolent, and can employ similar strategies as symbolic dissent, but
requires a mass mobilization component. Political violence is the use of
force by non-state actors to express discontent with the state. Across
those types of dissent, one should expect different responses. Some
dissent is less threatening than others. Nonviolent dissent poses little
physical threat to the state and offers no legitimate justification for
the use of repression, especially violent repression, to quell dissent.
Mass protest poses no direct physical threat to the state because it is
nonviolent, however it signals two dangers for the state: the issue is
sufficiently salient to generate dissent, and there is broad enough
support for the dissenting message to generate mass gatherings. However,
it is possible that malicious leaders that do not truly respect the
democratic norms of governance may view symbolic dissent and mass
protest as a threat to their own potential political success. In this
context the state is confronted with incentives to act, but the form of
those actions depends on what the state may legitimately do. Democracies
cannot legitimately use force or restrict civil liberties because of
such protest because they are rights protected by the democratic
constitutions. This leads to the first expectation of this project:
nonviolent dissent should never generate repressive responses from
democratic leaders. Because democracies promise civil liberties that
make nonviolent dissent possible, it would be illegitimate for
democratic leaders to restrict civil liberties or use violent repression
to quell nonviolent dissent. Any results to the contrary would suggest
that democratic leaders are not as willing to uphold civil liberties as
they claim. Hypothesis 1: Democracies will not use any type of
repression (intangible, tangible, or violent) in response to nonviolent
dissent (symbolic or mass). Political violence presents a threat to the
state and civilians, thus legitimizing the government use of force in
all regimes types and potentially increasing mass support of repressive
governing (Davis and Silver 2004). However, political violence is not a
carte blanche for violent repression in democracies. States have a
variety of tools that they should exhaust before resorting to any sort
of violent repression, and even in those cases where violent repression
is used it should be targeted and selective in order to minimize the
threat. A real-world example of a full democracy repressing in the face
of threat can be seen in France in 2016. In response to a string of
terrorist attacks claimed by ISIS, France declared a state of emergency
which restricts certain rights of citizens, and then proposed to change
and expand the powers elected officials have (Breeden 2016). A similar
example can be seen in the US response to 9/11 with the Patriot Act,
which expanded the power of the government to gather information about
the citizens that should be protected by the prerequisite of a warrant
to gather personal information (Baker 2015). In both cases, nonviolent
means were used to expand the reach of the state in favor of expanding
the state's ability to provide security. Research has also shown that
the population supports these changes as they feel threat levels
increasing (Davis and Silver 2004; Mondak and Hurwitz 2012).
Theoretically, expanding governmental powers and implementing nonviolent
repression via monitoring and information collection, detentions,
curfews, martial law, and declaring states of emergency as discussed
above allows the government to provide better security in a nonviolent
manner. At the same time, providing security in these ways violates the
civil liberties the state is required to defend to maintain its
democratic status in the normative sense, but it is unaffected by the
electoral definition implemented here. I expect that given sufficient
threat over time resulting from political violence democracies will
implement intangible and tangible repression first, and then violent
repression only in the face of extraordinarily threatening
circumstances.\\
Hypothesis 2: As political violence increases over time in democracies
leadership will be more likely to implement repressive policies to
ensure security.\\
The threat dissent poses accrues in memory over time. This factor has
not been accounted for empirically in the literature. Dissent at time t
may be independent in motivation and population from dissent at time
t-1, but the state recalls and is potentially threatened by both events
regardless of whether they are ideologically connected. More
importantly, theories of repression suggest that the frequency and
severity, in conjunction, are important in determining the states threat
perception (Davenport 1995). A state's leadership wants to survive
politically and protect the state as a whole. As a result, its threat
perception is a function of all recent previous dissent. In order to
test the temporal dynamics of state memory of dissent there must be an
explicit memory length to test. Theorizing about memory length is
arbitrary. Human recall is not infinite, nor is it absolutely confined
to a certain number of days or years. Empirically, Davenport (1996)
agnostically estimates a cross-correlation lag distribution analysis of
past years of dissent on current levels of repression and the data shows
that lagged dissent positively correlates and is statistically
significant up to seven years in the past. I will adopt a seven year
memory length in testing the temporal dynamics.\\
Another important aspect of memory is its decay. Individuals remember
what happened yesterday better than what happened last month. The
exception to this rule is when last month's event was especially
noteworthy. To account for this aspect of human memory, I must model the
decay of the impact of previous events. To account for this, I institute
a measure of dissent memory that incorporates a decay function over the
last seven years. This is discussed at greater length in the research
design section. Simply put: dissent that happened longer ago is
downwardly weighted in memory through a decay function that allows
particularly severe or intense events to have a longer-lasting presence
in the state's memory of dissent. The broad expectation as discussed
above is: as the memory of violent dissent increases, the state's
perceived level of threat increases, and repression becomes more likely.
Previous research analyzes regime types together, and here I analyze
democracies independently of nondemocracies. Thus, while this
expectation is not particularly novel in general, it is novel to expect
that this is a persistent effect in democracies. Once dissidents have
crossed the threshold to use violence the state has legitimacy in using
force to stop them for two reasons: they must stop them to provide
security for the general public and if they do not provide security they
face potential competency costs (Gelpi and Grieco 2015; Smith 1998). I
also expect that states will apply all possible repressive tools in the
context of extreme violent dissent because of the seriousness of the
threat political violence poses: states will do whatever possible to
address the security threat, and they will have legitimacy in doing so
because political violence is already occurring.

Data and Research Design I use the SPEED data aggregated at the monthly
level. My level of analysis is country-month. The SPEED data covers 192
countries, the Lexical Index covers 173 countries, and I include only
full democracies on the Lexical Index scale (6 out of 6). This
constrains my analysis 96 countries since World War II. The SPEED
dataset uses machine coded news articles to gather information about
dissent and repression events from around the world. It gathers
information from a variety of news sources and codes the types of
initiators (government or nongovernment), the types of events, the
number of participants, the length of event, the severity (number killed
and injured), and several other variables. Because the data is machine
generated, it lacks the preciseness that comes with human coding, but it
gains the advantage of having more observations at smaller time
intervals. The data in its original form has the event as the unit of
analysis with start and end dates specified. To account for dissent
events that lasted long periods of time I had to make some simplifying
assumptions. First, I expanded the data so that each event has one
observation for each month in which it was active at least one day.
Second, I have no choice but to assume that the overall repression and
dissent scores given in the SPEED data is constant over the entire
period that the event was active. For example, say that there was an
event with high levels of dissent that lasted six months, due to the
data's original format my only option is to assume that the dissent
level was equally high for each of the six months. This is the most
egregious assumption I make. After all events had one observation per
month in which they occurred, I collapsed all observations and took the
average levels of dissent and repression according to the types
discussed for each month. At this point, the unit of analysis becomes
country-month, and each dissent types has only one average value per
country-month. The dependent variable is a binary indicator of
repression. The dataset specifies three types of repression: intangible,
tangible, and violent. The dependent variables is coded zero if no
repression occurs, and one if the corresponding type of repression
occurs. I choose to binarize the data because I only really am concerned
with whether or not repression occurs, not how bad the repression was.\\
The primary independent variables are the memory of each type of
dissent. These variables are continuous. Symbolic ``small-gauge''
dissent is made up of four variables: whether the event was symbolic,
whether the event had more than ten initiators, the length of the event,
and whether violence is advocated . Mass dissent is a function of: the
number participating, whether a weapon was used, or injury was caused,
and whether violence was advocated. Political violence is a function of:
whether the attack was targeted at a person, an ordinal weapons grade
scale, and a count of those killed and injured. The memory variable was
crafted using an exponential decay function. I use the seven year memory
length suggested by Davenport (1996). The memory of dissent is equal to
dissent
memory\_mk=∑\emph{(y=1)\^{}7▒∑}(m=1)\textsuperscript{12▒〖dissent\_my}(〖(.9)〗\^{}y
) 〗. Where k is the temporal distance in terms of years and m is the
temporal distance in terms of months. For example, for the country-month
observation of January 2000, k=1 is the year of December 1999 to
December 1998, and each month in this year is m ϵ 1:12. Intuitively,
this function sums the values for all months in a year. Then, that sum
is exponentiated to the .9 to the y. For one year in the past y is equal
to one, and so on. Using January 2000 as an example still, the summed
dissent of a year seven years ago was taken to the power of
〖.9〗\^{}7=0.48. Assume the political violence has a value of 200 for
this month, its contribution to memory seven years later is
〖200〗\textsuperscript{(〖.9〗}7 )=12.61. This function downwardly
weights the events of the past, as they become less recent up to seven
years, at which point they are dropped from memory. Relevant controls
include: logged GDP per capita in constant 2011 US Dollars (The Next
Generation of the Penn World Table 2013), total population, urban
population(Coppedge et al. 2011) , and whether or not the state was
currently experiencing a war according to the Correlates of War
dataset(Sarkees, Sarkees, and Wayman 2010). Economic development has
been hypothesized to decrease the likelihood of repression (Henderson
1991). States in wars have been shown to repress more often as a way to
secure the domestic front (Young 2013). Because my dependent variable is
binary, I will use logistic regression to test my hypotheses. The unit
of analysis is country-month, but some variables vary at the
country-year unit of analysis. To deal with this multilevel structure of
the data I use random effects models using the year as the random
intercept variable. This corrects standard errors that would have been
misspecified using standard logistic regression and allows confident
hypothesis testing. To account for time I implement Carter and
Signorino's (2010) advice and use a time polynomial.\\
To account for the highly correlated nature of my dissent variables I
estimate them in separate models. I also estimate a fourth model for
each dependent variable in which the three dissent variables are added
together. Each type of repression represents its own dependent variable,
so there are three sets of four estimations. It is logical to assume
that if higher levels of dissent such as political violence or mass
protests are occurring then lower levels of dissent are also occurring.
However, because these variables are highly correlated they cannot be
included in the same model. In the fourth model the summed memory of
dissent presents the total memory of all dissent types added together.
The results below are coefficient plots of the standardized memory
variables . There are tables including full results with control
variables in the appendix, but for the main text of the manuscript
control variables are excluded from discussion. Results and Discussion
Figure 1 below shows the coefficient plot of the main independent
variables. Each dot represents a coefficient from an independent
multilevel logistic regression. I estimate each model separately to
avoid multicollinearity problems associated with estimating the
variables separately. This way, the variables do not interfere with each
other and allow me to interpret them independently.

\begin{verbatim}
First, I will interpret the effect that symbolic dissent has on the likelihood of repression. The results strongly contradict my hypothetical expectation the democracies do not repress in response to symbolic dissent. Symbolic dissent should represent the strongest protected class of civil liberties that democratic citizens have as freedom of speech and press are fundamental civil liberties that underpin democratic society. The results show that as the memory of symbolic dissent increases the likelihood of all types of repression increases. Particularly troublesome for the prospect of democratic governance is that disproportionate response to symbolic dissent is apparently taking place because tangible and violent repression are in no way ‘responding in kind’ to symbolic dissent. 
One concern with this finding could be the distribution of the data. Indeed, the data is right-skewed with most of the observations being near zero and few observations being near the maximum. If this were the problem one common solution to handling variables of this distribution type is to log them and include this transformed variable in the final model. This specification returns similar results. In fact, this transformed specification returns similar results for all variable types because right-skewed distributions are common with this data. 
One finding the bodes well for democratic optimism is that mass protests do not beget repression of any kind. These rights are also generally protected by constitutions in democracies and the public should not be punished repressively for exercising these rights. This expectation is supported with the data, indicating the democracies are not willing to use intangible, tangible, or violent repression as a response to dissent. 
Hypothesis 2’s expectation is upheld. As the memory of political violence increases the state is more likely to use all types of repression to combat it. This is a sign of good governance. Political violence poses the greatest threat to the population, and the only situation in which it is remotely justifiable for a democratic state to resort to any type of repressive policies. In this context, democracies do respond in kind, however I cannot conclude that they are targeting their repressive policies to the perpetrators. This finding may bode poorly for the overall prospect of freedoms in democracy during times of political violence: if threat is sufficiently high everybody suffers in the name of public security. This finding is corroborated with the additive measure of total dissent: when total dissent in memory increases, and the threat it poses therefore increases, all types of repression are more likely to occur. 
These results seem to suggest that democracies willingly repress when circumstances legitimate it, and that they repress when they think nobody is watching. Political violence legitimates the use of repressive policies, so democratic leaders face no cost for enacting these policies. Mass protests pose little threat to public security, but great threat leadership survival. Mass protests are also generally more organized and state response to such events are much more visible than symbolic dissent because of the organized nature of the dissent. In other words, if the state enacts repressive policies to fight symbolic dissent then they can get away with it unnoticed if they use intangible repression. Few would notice, and therefore few would care. However, it makes little sense that democracies would use more severe repressive policies than intangible repression as a response to symbolic dissent. 
\end{verbatim}

Conclusion I argue that democratic leaders use repression as a policy
tool with the threat level of the state legitimizes the use of
repressive policies. I use the SPEED dataset to break apart repression
into three categories: intangible, tangible, and violent. I also break
apart dissent into three categories: symbolic, mass, and violent. I
provide a more nuanced look at the dissent-repression nexus not only by
looking at a disaggregated form of repression and dissent, but also
looking at exclusively democracies and looking at the memory of dissent
as opposed to exclusively contemporaneous dissent. I find that
democracies respond with all types of repression as a response to
symbolic dissent and political violence, and do not respond to mass
dissent with repression. I think this finding is explained by looking at
the visibility of state actions. Mass dissent is a very visible event,
especially when it is persistent and large-scale. Therefore, repressing
it is very visible to the public and poses high costs in terms of
democratic legitimacy for infringing upon rights of expression promised
by democratic states. Symbolic dissent, however, is possible to repress
with relatively few people noticing via intangible repression. I leave
it up to future avenues of work to understand why symbolic dissent
begets tangible and violent repression. I also find that democracies
response to political violence with all types of repression as it grows
in memory. This is because it poses the greatest threat to security of
the state and legitimizes the use of repression.   Appendix Table A1:
Dissent Memory and Intangible Repression M1 M2 M3 M4 DV: Intangible
Repression Symb. Dissent Memory 4.453*\\
(0.589)\\
Mass Dissent Memory 1.647\\
(0.910)\\
Political Violence Memory 3.598*\\
(0.448) Total Dissent Memory 4.178\emph{ (0.466) War 0.254} 0.355*
0.270* 0.249\emph{ (0.0741) (0.0723) (0.0734) (0.0736) Urban Population
1.73e-08} 7.80e-09 1.35e-08* 1.45e-08\emph{ (5.58e-09) (6.20e-09)
(5.24e-09) (5.20e-09) Total Population -4.53e-09} -2.56e-09 -4.49e-09*
-4.25e-09\emph{ (1.96e-09) (2.13e-09) (1.82e-09) (1.84e-09) Logged GDP
Per Capita -0.0315 0.297 -0.0847 -0.114 (0.158) (0.192) (0.146) (0.144)
Time 0.00127 0.00129 -0.00107 -0.000690 (0.00511) (0.00520) (0.00520)
(0.00515) Time Squared -0.00000246 0.000000979 -0.000000371 -0.000000639
(0.0000136) (0.0000137) (0.0000138) (0.0000137) Time Cubed -1.23e-09
-6.48e-09 -9.15e-10 -6.91e-10 (1.10e-08) (1.11e-08) (1.12e-08)
(1.11e-08) Constant -5.444} -8.126* -4.486* -4.398\emph{ (1.339) (1.587)
(1.266) (1.244) Observations 27672 27672 27672 27672 Standard errors in
parentheses } p \textless{} .05

  Table A2: Dissent Memory and Tangible Repression M5 M6 M7 M8 Tangible
Repression Symb. Dissent Memory 4.404*\\
(0.465)\\
Mass Dissent Memory 0.658\\
(0.678)\\
Political Violence Memory 3.631*\\
(0.331) Total Dissent Memory 3.970\emph{ (0.363) War 0.317} 0.398*
0.296* 0.298\emph{ (0.0638) (0.0626) (0.0640) (0.0637) Urban Population
2.45e-09 -1.15e-08} -2.56e-09 3.09e-10 (4.76e-09) (5.39e-09) (4.86e-09)
(4.60e-09) Total Population -7.49e-10 2.62e-09 -4.83e-10 -7.19e-10
(1.65e-09) (1.88e-09) (1.67e-09) (1.59e-09) Logged GDP Per Capita 0.0494
0.305 -0.0163 -0.0196 (0.129) (0.159) (0.124) (0.120) Time 0.0120*
0.0125* 0.00927* 0.00957\emph{ (0.00386) (0.00397) (0.00393) (0.00388)
Time Squared -0.0000288} -0.0000264* -0.0000263* -0.0000257\emph{
(0.0000100) (0.0000102) (0.0000102) (0.0000101) Time Cubed 2.06e-08}
1.63e-08* 2.12e-08* 2.00e-08\emph{ (7.98e-09) (8.05e-09) (8.10e-09)
(8.02e-09) Constant -6.543} -8.729* -5.484* -5.577\emph{ (1.065) (1.271)
(1.036) (1.010) Observations 27672 27672 27672 27672 Standard errors in
parentheses } p \textless{} .05

  Table A3: Dissent Memory and Violent Repression M9 M10 M11 M12 Violent
Repression Symb. Dissent Memory 4.884*\\
(0.444)\\
Mass Dissent Memory 1.272\\
(0.715)\\
Political Violence Memory 4.588*\\
(0.317) Total Dissent Memory 4.809\emph{ (0.341) War 0.299} 0.436*
0.242* 0.253\emph{ (0.0637) (0.0614) (0.0646) (0.0642) Urban Population
1.49e-08} 9.58e-10 9.78e-09* 1.21e-08\emph{ (4.94e-09) (5.31e-09)
(4.49e-09) (4.58e-09) Total Population -4.20e-09} -1.15e-09 -3.94e-09*
-3.95e-09\emph{ (1.72e-09) (1.85e-09) (1.56e-09) (1.61e-09) Logged GDP
Per Capita -0.445} -0.192 -0.510* -0.517\emph{ (0.124) (0.144) (0.109)
(0.112) Time 0.000120 -0.000105 -0.00291 -0.00216 (0.00369) (0.00376)
(0.00376) (0.00371) Time Squared 0.00000545 0.0000117 0.00000703
0.00000711 (0.00000982) (0.00000988) (0.00000997) (0.00000988) Time
Cubed -8.18e-09 -1.59e-08} -6.11e-09 -7.09e-09 (7.96e-09) (7.96e-09)
(8.08e-09) (8.01e-09) Constant -1.427 -3.504* -0.237 -0.423 (0.988)
(1.128) (0.891) (0.910) Observations 27672 27672 27672 27672 Standard
errors in parentheses * p \textless{} .05

  References Baker, Peter. 2015. ``In Debate Over Patriot Act, Lawmakers
Weigh Risks vs.~Liberty.'' The New York Times.
\url{https://www.nytimes.com/2015/06/02/us/politics/in-debate-over-patriot-act-lawmakers-weigh-risks-vs-liberty.html}
(April 8, 2018). Bloch-Elkon, Yaeli. 2011. ``The Polls---Trends: Public
Perceptions and the Threat of International Terrorism after 9/11.''
Public Opinion Quarterly 75(2): 366--92. Breeden, Aurelien. 2016.
``France Weighs Limits of Liberty, Equality and Citizenship.'' The New
York Times.
\url{https://www.nytimes.com/interactive/2016/02/16/world/europe/france-constitution-new-laws.html},
\url{https://www.nytimes.com/interactive/2016/02/16/world/europe/france-constitution-new-laws.html}
(April 8, 2018). Bueno de Mesquita, Bruce, Alastair Smith, Randolph M.
Siverson, and James D. Morrow. 2003. The Logic of Political Survival.
Cambridge: MIT Press. Carter, David B., and Curtis S. Signorino. 2010.
``Back to the Future: Modeling Time Dependence in Binary Data.''
Political Analysis 18(3): 271--92. Dahl, Robert A. 1989. Democracy and
Its Critics /. New Haven: Yale University Press. Danneman, Nathan, and
Emily Hencken Ritter. 2014. ``Contagious Rebellion and Preemptive
Repression.'' Journal of Conflict Resolution 58(2): 254--79. Davenport,
Christian. 1995. ``Multi-Dimensional Threat Perception and State
Repression: An Inquiry into Why States Apply Negative Sanctions.''
American Journal of Political Science 39(3): 683--713. ---------. 1996.
``The Weight of the Past: Exploring Lagged Determinants of Political
Repression.'' Political Research Quarterly 49(2): 377--403. ---------.
1999. ``Human Rights and the Democratic Proposition.'' Journal of
Conflict Resolution 43(1): 92--116. ---------. 2007. ``State Repression
and Political Order.'' Annual Review of Political Science 10(1): 1--23.
Davenport, Christian, and David A. Armstrong. 2004. ``Democracy and the
Violation of Human Rights: A Statistical Analysis from 1976 to 1996.''
American Journal of Political Science 48(3): 538--54. Davis, Darren W.,
and Brian D. Silver. 2004. ``Civil Liberties vs.~Security: Public
Opinion in the Context of the Terrorist Attacks on America.'' American
Journal of Political Science 48(1): 28--46. Finseraas, Henning, and Ola
Listhaug. 2013. ``It Can Happen Here: The Impact of the Mumbai Terror
Attacks on Public Opinion in Western Europe.'' Public Choice 156(1/2):
213--28. Francisco, Ronald A. 1995. ``The Relationship between Coercion
and Protest: An Empirical Evaluation in Three Coercive States.'' The
Journal of Conflict Resolution 39(2): 263--82. ---------. 1996.
``Coercion and Protest: An Empirical Test in Two Democratic States.''
American Journal of Political Science 40(4): 1179--1204. Franklin, James
C. 2008. ``Shame on You: The Impact of Human Rights Criticism on
Political Repression in Latin America.'' International Studies Quarterly
52(1): 187--211. Gelpi, Christopher, and Joseph M. Grieco. 2015.
``Competency Costs in Foreign Affairs: Presidential Performance in
International Conflicts and Domestic Legislative Success, 1953--2001.''
American Journal of Political Science 59(2): 440--56. Gerschewski,
Johannes. 2013. ``The Three Pillars of Stability: Legitimation,
Repression, and Co-Optation in Autocratic Regimes.'' Democratization
20(1): 13--38. Gibney, Mark et al. 2018. ``The Political Terror Scale.''
ht�tp://www.polit�ic�al�ter�rorscale.org (March 31, 2019). Haggard,
Stephan, and Robert R. Kaufman. 2016. Dictators and Democrats: Masses,
Elites, and Regime Change /. Princeton: Princeton University Press.
Henderson, Conway W. 1991. ``Conditions Affecting the Use of Political
Repression.'' Journal of Conflict Resolution 35(1): 120--42. ---------.
1993. ``Population Pressures and Political Repression.'' Social Science
Quarterly (University of Texas Press) 74(2): 322--33. Lichbach, Mark
Irving. 1987. ``Deterrence or Escalation? The Puzzle of Aggregate
Studies of Repression and Dissent.'' The Journal of Conflict Resolution
31(2): 266--97. Mitchell, Neil J., and James M. McCormick. 1988.
``Economic and Political Explanations of Human Rights Violations.''
World Politics 40(4): 476--98. Mondak, Jeffery J., and Jon Hurwitz.
2012. ``Examining the Terror Exception Terrorism and Commitments to
Civil Liberties.'' Public Opinion Quarterly 76(2): 193--213. Moore, Will
H. 1998. ``Repression and Dissent: Substitution, Context, and Timing.''
American Journal of Political Science 42(3): 851--73. ---------. 2000.
``The Repression of Dissent: A Substitution Model of Government
Coercion.'' The Journal of Conflict Resolution 44(1): 107--27. Nardulli,
Peter F., Scott L. Althaus, and Matthew Hayes. 2015. ``A Progressive
Supervised-Learning Approach to Generating Rich Civil Strife Data:''
Sociological Methodology.
\url{https://journals.sagepub.com/doi/pdf/10.1177/0081175015581378}
(November 20, 2018). Ober, Josiah. 2012. ``Democracy's Dignity.'' The
American Political Science Review 106(4): 827--46. Poe, Steven C., and
C. Neal Tate. 1994. ``Repression of Human Rights to Personal Integrity
in the 1980s: A Global Analysis.'' American Political Science Review
88(4): 853--72. Regan, Patrick M., and Errol A. Henderson. 2002.
``Democracy, Threats and Political Repression in Developing Countries:
Are Democracies Internally Less Violent?'' Third World Quarterly 23(1):
119--36. Sarkees, Meredith Reid, Meredith Sarkees, and Frank Wayman.
2010. Resort to War, 1816-2007. Washington: CQ Press. Smith, Alastair.
1998. ``International Crises and Domestic Politics.'' The American
Political Science Review 92(3): 623--38. Stohl, Michael, David Carleton,
George Lopez, and Stephen Samuels. 1986. ``State Violation of Human
Rights: Issues and Problems of Measurement.'' Human Rights Quarterly
8(4): 592--606. Taylor, Charles, and David Jodice. 1983. World Handbook
of Political and Social Indicators III. New Haven: Yale University
Press. The Next Generation of the Penn World Table. 2013. Cambridge,
Mass: National Bureau of Economic Research. Young, Joseph K. 2013.
``Repression, Dissent, and the Onset of Civil War.'' Political Research
Quarterly 66(3): 516--32.

\end{frame}

\end{document}
